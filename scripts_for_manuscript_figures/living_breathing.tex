\documentclass[11pt, oneside]{article}   	% use "amsart" instead of "article" for AMSLaTeX format
\usepackage{geometry}                		% See geometry.pdf to learn the layout options. There are lots.
\geometry{letterpaper}                   		% ... or a4paper or a5paper or ... 
%\geometry{landscape}                		% Activate for rotated page geometry
%\usepackage[parfill]{parskip}    		% Activate to begin paragraphs with an empty line rather than an indent
\usepackage{graphicx}				% Use pdf, png, jpg, or eps§ with pdflatex; use eps in DVI mode
								% TeX will automatically convert eps --> pdf in pdflatex		
\usepackage{amssymb}
\usepackage{verbatim}

%SetFonts

%SetFonts


\title{Post Wedell-Park Directions}
\author{George Chacko}
%\date{}							% Activate to display a given date or no date

\begin{document}
\maketitle
%\section{}
%\subsection{}

\section*{Overview}

A number of directions are obvious in terms of developing the Wedell-Park discoveries. We note in the Conclusions that breaking larger clusters and considering non-disjoint approaches is a priority (need to check if there is anything else).
With respect to the first, Akhil identified a single node (id=2479629) apparently with 29,353 references. The id resolves to,\\

\par
\noindent
@book{Zimmermann2016, \\
  doi = {10.1007/978-3-319-26587-2}, \\
  url = {https://doi.org/10.1007/978-3-319-26587-2}, \\
  year = {2016}, \\
  publisher = {Springer International Publishing}, \\
  author = {Arthur Zimmermann}, \\
  title = {Tumors and Tumor-Like Lesions of the Hepatobiliary Tract} \\
} \\

\noindent which appears to be a medical textbook published more than once and with many chapters. Thus, the number of references is plausible. 

This observations raises the question of `high-referencing' articles that I first read about in  Henry Small's 1984 article (Small, Henry, and Ernest Sweeney. "Clustering thescience citation index® using co-citations." Scientometrics 7.3 (1985): 391-409.).  Upon investigation, I find that the exosome edgelist consisting of  14,695,475 nodes (99,663,372 edges)  contains 667 nodes with more than 1,000 references (out-degree) each and 2,205 nodes with more than 500 references each.

It is very likely that such high-referencing articles nucleate clusters in our total degree approach and while high-citing is legitimate, high-referencing is likely to generate large clusters that may not indicate research activity so much as the presence of high-referencing articles (typically reviews/surveys)

A simple approach might be to remove all articles with more than `n' references. I propose 500 (although we might even consider 250). When such a filter is applied, all edges with 2,205 nodes are deleted. The exosome network now consists of 14,300,179 nodes and 96,979,448 edges. Some extra nodes are lost since their only edges involve the 2,205 nodes.

\begin{verbatim}
library(data.table)
x <- fread('citing_cited_network.integer.tsv')

supra_499refs_vec <- x[,.N,by='V1'][N>= 500][,V1]
> print(length(supra_499refs_vec))
[1] 2211

infra_500refs <- x[!(V1 %in% supra_499refs_vec)][!(V2 %in% supra_499refs_vec)]
>  dim(infra_500refs)
[1] 96975240        2

# similarly 
supra_249refs_vec <- x[,.N,by='V1'][N>= 250][,V1]
infra_250refs <- x[!(V1 %in% supra_249refs_vec)][!(V2 %in% supra_249refs_vec)]
> dim(infra_250refs)
[1] 92186040        2
\end{verbatim}

\section*{Proposed Initial Experiments}
\begin{itemize}
\item Run  IKC\_k10\_p2 on the  infra\_250refs and infra\_500refs files
\item Run the these two files through the full pipeline using IG
\item Run the these two files through the full pipeline using RG
\end{itemize}

\end{document}  